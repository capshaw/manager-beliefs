\documentclass[11pt]{amsart}
\usepackage[letterpaper]{geometry}
\geometry{margin=1in, includehead,includefoot}

\usepackage[parfill]{parskip} 
\usepackage{graphicx}
\usepackage{amssymb}
\usepackage{epstopdf}
\usepackage{tabularx}
\usepackage{makecell}
\usepackage{enumitem}
\usepackage{multicol}
\usepackage{hyperref}
\usepackage{fancyhdr}

\pagestyle{fancy}% Change page style to fancy
\fancyhf{}% Clear header/footer
\fancyhead[R]{\thepage}
\fancyhead[L]{\title}
\renewcommand{\headrulewidth}{0.4pt}% default is 0pt
\renewcommand{\footrulewidth}{0pt}% default is 0pt

\hypersetup{
    colorlinks=true,
    linkcolor=blue,
    filecolor=blue,      
    urlcolor=blue,
}

\title{My beliefs as a manager}
\author{Andrew Capshaw}

\begin{document}

\thispagestyle{empty}

% - - - - - - - - - - - - - - - - - - - - - - - - - - - - - - - - - - - - - - - - - - - - - - - - - - - - - - - - - - - - - - - - - - - - - - - - - - - - - - - - - - - - - - - - - - 
% Heads up: this is super messy. At least the PDF looks nice.
% - - - - - - - - - - - - - - - - - - - - - - - - - - - - - - - - - - - - - - - - - - - - - - - - - - - - - - - - - - - - - - - - - - - - - - - - - - - - - - - - - - - - - - - - - - 

% TODO: degree of belief (strongly, weekly held)
% TODO: priors
% TODO: on teams and their structure and working together / people over process

\maketitle

This document is intended to capture my beliefs as a manager. The audience of this document could be my direct reports, my manager, or even my future managers. It's intended to be a starting place for conversation about expectations, not an immutable manuscript. As a reader, let me know what you think and let's talk! 

Like my views on management, this document is a continuous work-in-progress and changes often.

\section*{I'm playing support... for everyone on my team} % Shield
\subsection*{I am here to serve my team}
The phrase `servant leadership’ is a popular one among managers. I prefer the phrase `service-oriented leadership'. Regardless, the meaning is the same -- I'm here to serve those on my team. I mean it. My sole purpose as a manager is to elevate my engineers and the team to be happy and successful. I exist to remove their roadblocks, celebrate their wins, and assist them in fulfilling their career ambitions, among other things. Without a team to support, I am unnecessary.

\begin{quote}
The manager’s function is not to make people work, but to make it possible for people to work.

-- Tom DeMarco, \emph{Peopleware}
\end{quote}

\subsection*{I am only successful when my team is successful}
This is a corollary to the previous. As a manager, I should be measured by the success of my direct reports and team as a whole. Keeping my team  happy with their work life -- with the support needed to be productive -- is my number one priority. Anything that gets in my team's way of being happy and productive is where I should focus my efforts.

\begin{quote}
[T]he performance rating of a manager cannot be higher than the one we would accord to (their) organization

-- Andrew Grove, \emph{High Output Management}
\end{quote}

\subsection*{I will always be equitable and inclusive}
I will treat everyone equally. I will treat my engineers with respect. Diversity makes us stronger. I will hold the team accountable for being inclusive. 

\subsection*{I'm open to being wrong and to all feedback, big or small}
I am happy for my team to challenge me and my ways. It’s understandable if when one wants to hold back at times. It’s natural when there is a perceived power imbalance. When and where one is comfortable being open, open communication allows me to focus, iterate, and improve.

\subsection*{I am against process that only serves me}
Sometimes process exists only for the sake of the manager, especially process around reporting. If I’m asking my team to do something out of their way just so I can measure or report on something, it may not be in the best interest of the team.

% - - - - - - - - - - - - - - - - - - - - - - - - - - - - - - - - - - - - - - - - - - - - - - - - - - - - - - - - - - - - - - - - - - - - - - - - - - - - - - - - - - - - - - - - - - 

\section*{On Coaching} % Ladder

\subsection*{I will give my team continuous feedback often} 
Evaluation feedback should not be infrequent or greatly surprising. If an engineer on my team is surprised during an evaluation period, I’ve done a poor job of communicating and collaborating on their performance. I expect myself to check-in with my team frequently and communicate clearly how they are doing.

\subsection*{I will give my team opportunities to grow}
It can be frustrating as an engineer to feel stagnant. I will give my team actionable feedback to achieve their goals and assist them in finding opportunities to demonstrate those behaviors.

\subsection*{I’ll support career transitions if an engineer wants to make a move}
I want my team to feel comfortable discussing a potential future not on my team or in a different role if there are ways they want to grow that are not supported by their current team. I’ll support engineers and help them grow, regardless of whether it’s possible on my team or not.

\subsection*{I will celebrate my engineer's wins how they see fit and never provide constructive criticism publicly}
I will always defer to an individual's level of comfort when celebrating wins. I’d love to share them publicly, but If the engineer doesn't want to share them publicly, I won’t! I will never provide constructive criticism about an engineer's performance publicly – I see that as toxic.

\subsection*{I will give my engineers my full attention in 1-on-1s}
It’s incredibly easy to have competing priorities as a manager. 1-on-1s are the report's meeting. They’re dedicated time for us to talk about whatever topics an engineer desires. It would be rude of me as a manager to be distracted during this time. If for some reason there is a production fire or other interruption that prevents us from having a meaningful 1-on-1, I’d prefer to reschedule over only partially being able to pay attention to my team.

% - - - - - - - - - - - - - - - - - - - - - - - - - - - - - - - - - - - - - - - - - - - - - - - - - - - - - - - - - - - - - - - - - - - - - - - - - - - - - - - - - - - - - - - - - - 

\section*{On Collaboration}

\subsection*{I will lead by example}
I will not ask my team to exemplify behaviors that I do not show. If I want my team to have high ownership, I'll exemplify that ownership. If I want my team to be accountable for their word, I'll be accountable for my word. Somewhat related, I'll join in frontline engineering work where appropriate, while also acknowledging that it's easy to get in the way as a manager due to time constraints (it's a balance).

\subsection*{I will protect my team from unnecessary requests}
One function of a manager is to protect their direct reports from incoming requests that are not high priority. I will protect my team from requests that don’t serve to move our team towards its goals. When possible, I will gather as much information about a request and its purpose before bringing it to the team so that there are meaningful actions we can take on the request.

\subsection*{I will be transparent everywhere I can be}
By default, if I can be transparent, I will be. Hiding information from my team does not make the team better. I will properly qualify when I don’t know an answer fully.

% TODO: refine this
% \subsection{I will let you have the impactful work and take the support work myself}
% There will always be some amount of ‘keep the lights on’ or cleanup work that needs to be done. It’s not particularly interesting or glamorous most of the time. When I have opportunities to take on engineering work myself, I’ll default to leaving the interesting and impactful things for the team and take the work is uninteresting but needs doing. This might be a slightly controversial opinion – I think some managers would think it’s healthy for everyone to have a balance of work. For me, this is an acknowledgement that high achievers who are aligned with product impact usually want to be working on things that are meaningful. It’s also an acknowledgement that managers work on an inefficient schedule to write solid code (i.e. the manager’s schedule). If I can play necessary support and let you have the work that you enjoy doing, it’s a win-win situation.

% \subsection{I am not dogmatic about team process}
% TODO

% - - - - - - - - - - - - - - - - - - - - - - - - - - - - - - - - - - - - - - - - - - - - - - - - - - - - - - - - - - - - - - - - - - - - - - - - - - - - - - - - - - - - - - - - - - 

% \section{On product management}

% \subsection{Engineering is a product role}
% TODO Our job is to build a product.

% \subsection{Empathy for end user}
% TODO

% \subsection{Measurement, but only the right things}
% TODO

% \subsection{Setting team direction is together thing}
% TODO

% - - - - - - - - - - - - - - - - - - - - - - - - - - - - - - - - - - - - - - - - - - - - - - - - - - - - - - - - - - - - - - - - - - - - - - - - - - - - - - - - - - - - - - - - - - 

\section*{Parting Thoughts}
My promise as a manager is that I will strive to do these things and more. I am always open to being wrong. I will never stop growing and improving myself. This document will grow and change over time, like me.

% - - - - - - - - - - - - - - - - - - - - - - - - - - - - - - - - - - - - - - - - - - - - - - - - - - - - - - - - - - - - - - - - - - - - - - - - - - - - - - - - - - - - - - - - - - 

\section*{Influences}

I intend to flesh this section out a bit more with the why and how, but here are a few books that have influenced my way of thinking as a manager.

Listed from most to least influential on my beliefs.

\begin{itemize}
  \item \emph{Radical Candor} by Kim Malone Scott 
  \item \emph{Peopleware} by Tom DeMarco
  \item \emph{High Output Management} by Andrew Grove
\end{itemize}

\end{document}  